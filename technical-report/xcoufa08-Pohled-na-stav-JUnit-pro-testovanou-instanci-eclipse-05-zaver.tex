%=========================================================================%
% - - - KAPITOLA 5: Z Á V Ě R                                             %
\chapter{Závěr}                                                           %
%=========================================================================%
Tato práce obsahuje návrh a implementaci nástrojů poskytujících informace o~probíhajících testech uživatelského rozhraní platformy Eclipse. Tyto nástroje  jsou navrženy pro funkci s~testovacím rámcem JUnit, obsaženým v~Eclipse JDT (\emph{Java Development Tools}). V~první části je popsána architektura platformy Eclipse, se zaměřením na strukturu a tvorbu zásuvných modulů. Druhá část se zabývá architekturou testovacího rámce JUnit, jeho aplikačním rozhraním, možnostmi rozšíření a jeho integrací ve vývojovém prostředí Eclipse.

Hlavní část práce představuje autorem implementovaný nástroj TRV, skládající se ze dvou částí\,--\,InRunJUnit a TRView. Díky těmto nástrojům lze sledovat průběh testů uživatelského rozhraní, bez problémů s~aktivitou okna. Uživateli je zobrazena stromová struktura s~vyznačenými stavy jednotlivých testovacích případů. Zároveň poskytuje uživateli přehled o~počtu běhů testů, testových chyb, běhových chyb a ignorovaných testovacích případů. V~případě chybného testovacího případu zobrazuje i výpis posledních volání na zásobníku (\emph{angl. stack trace}).

Nástroj TRV byl předán firmě Red Hat, kde bude diskutováno a navrženo jeho budoucí použití. Primární nasazení by mělo spočívat ve spolupráci se serverem Jenkins. Dále je zde možnost integrace nástroje TRV do testovacího rámce RedDeer, který již poskytuje některé nástroje pro CI (\emph{angl. continuous integration}).