%=========================================================================%
% KAPITOLA 1: Ú V O D                                                     %
\chapter{Úvod}                                                            %
%=========================================================================%
V~dnešní době se na testování klade velký důraz, a proto je potřeba se touto částí vývoje software zabývat detailně. Aplikace lze testovat různými způsoby, od jednoduchých manuálních testů po sofistikované nástroje automaticky spouštějící vybrané sady testů. Pro tyto účely existuje mnoho nástrojů\footnote{\url{https://en.wikipedia.org/wiki/List_of_unit_testing_frameworks}}, které usnadňují programátorům práci. Cílem těchto nástrojů je redukce množství napsaného kódu opakujícího se v~testech pro podobné komponenty nebo jejich vlastnosti.

Co se týče programovacího jazyka Java, můžeme vybírat z~velkého množství nástrojů\footnote{\url{https://en.wikipedia.org/wiki/List_of_unit_testing_frameworks\#Java}} pro testování podle toho, jaký aspekt software chceme testovat. Pro testování Servler, Bean a Java tříd lze testovat pomocí nástrojů jako jsou například \emph{Servlets}, \emph{JUnit}, \emph{Arquillian}, \emph{ServletUnit} nebo \emph{Mock objects}. Pro testování grafického uživatelského rozhraní vytvořeného pomocí Swing lze použít například \emph{UISpec4j}, \emph{Abbot}, \emph{Fest}, \emph{QF-Test} a další. Pro funkcionální testování lze použít například \emph{HTTPUnit}, \emph{JWebUnit}, \emph{TestNG} nebo \emph{Selenium Webdriver}, zatímco pro výkonnostní testování lze použít například \emph{Apache JMeter} \cite{softwaretestinghelp}.

Tato práce se zabývá popisem infrastruktury vývojového prostředí \emph{Eclipse} (dále zkráceně Eclipse IDE) a jeho zásuvných modulů s~přihlédnutím k~budoucímu použití pro vytvoření vlastního zásuvného modulu. Dále se zabývá popisem testovacího rámce JUnit a jeho architekturou. JUnit je snadno rozšiřitelný a je obsažen ve velkém množství testovacích nástrojů a vývojových prostředí včetně Eclipse IDE, kde ho lze použít při testování jeho grafického uživatelského rozhraní (dále zkráceně GUI). Klíčovou částí práce je návrh a implementace zásuvného modulu pro Eclipse IDE, který zpracovává data z~testovacího rámce JUnit a posílá informace o~průběhu testů do klientské aplikace, kde je zobrazuje uživateli. Důvodem pro tvorbu tohoto projektu je potřeba programátora zjistit v~jaké fázi se probíhající sada testů nachází, který test v~dané chvíli běží a jak dopadly již proběhlé testy.

Podobným již existujícím nástrojem umožňující zobrazení informací o průběhu testů je \emph{pohled JUnit}. Ten je bohužel při testování GUI překryt oknem s testovanou instancí a v případě přepnutí na okno s pohledem JUnit riskujeme ztrátu aktivity okna a selhání testů. Navíc při testování grafického uživatelského rozhraní hraje roli spousta dalších proměnných. Často jen změna komponenty nebo konfigurace platformy, na které testujeme, může způsobit selhání testů. Proto je návrh implementované aplikace přizpůsoben co nejmenšímu zásahu do GUI testované instance Eclipse IDE.
